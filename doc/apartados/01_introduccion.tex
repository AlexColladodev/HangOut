\chapter{Introducción}
\section{Contexto y motivación}

Se estima que el 60 \% de los estudiantes de grado de la Universidad de Granada provienen de otras ciudades de
España o del extranjero \cite{ugr}. Muchos de ellos llegan por primera vez careciendo de información básica sobre qué
establecimientos visitar y qué hacer durante su estancia. Son personas jóvenes con intenciones de crear vínculos
y grupos con otros sobre sus aficiones y gustos compartidos.

Además de ser una ciudad universitaria, Granada es una ciudad histórica que atrae a muchos turistas que vienen
por primera vez con la emoción de descubrir este rincón de España. Estos turistas enfrentan el obstáculo de
tener que buscar entre todos los sitios disponibles hasta encontrar el más adecuado a sus preferencias.

Esta situación no es exclusiva de Granada; es un problema común en muchas ciudades del mundo, donde la falta de
información sobre qué hacer o dónde ir se convierte en un desafío habitual.

La elección de este tema surge de mi propia experiencia al llegar a Granada por primera vez. En un entorno nuevo
y desconocido, sin conocer a nadie y habiendo hecho amigos en la universidad, aún no sabía a qué lugares ir,
cuál era el ambiente de esos sitios ni qué tipo de personas los frecuentaban.

Frente a este escenario, el propósito del proyecto es el desarrollo de una aplicación móvil que asista a las
personas en la búsqueda de lugares o eventos de ocio, ofreciendo una lista de establecimientos y eventos. De
esta forma, se busca garantizar una experiencia de usuario óptima, permitiendo a los usuarios explorar
descripciones detalladas de los establecimientos según el ambiente que ofrecen, y realizar búsquedas
personalizadas según sus preferencias, sin necesidad de revisar individualmente cada página web de los
establecimientos.


\section{Objetivos}

Dado este enfoque, se han analizado y planteado una serie de objetivos para la creación de una aplicación móvil
que ofrezca una solución nueva y eficiente para la gestión y búsqueda de establecimientos orientados al ocio. El
objetivo principal es proporcionar al usuario una herramienta que facilite este proceso.

A continuación, se presentarán los objetivos principales de la aplicación, los cuales serán descritos en detalle
a lo largo del proyecto:

\begin{enumerate}
    \item \textbf{Registro:} Que los usuarios y administradores de establecimientos puedan registrarse con sus
          credenciales personales y únicas.
    \item \textbf{Facilitar la Búsqueda de Establecimientos:} Proporcionar un sistema que permita a los usuarios
          encontrar establecimientos de ocio según sus preferencias.
    \item \textbf{Gestión de Eventos y Ofertas:} Permitir la creación, borrado y modificación de eventos y
          ofertas para que los establecimientos puedan anunciar información que le interese al usuario.
    \item \textbf{Interacción Social:} Los usuarios pueden seguir a otros, crear actividades y organizar eventos
          grupales privados.
    \item \textbf{Reseñas a Establecimientos:} Los usuarios pueden calificar la experiencia en un
          establecimiento dejando reseñas con una calificación y un mensaje
    \item \textbf{Interfaz Intuitiva:} Desarrollar una interfaz de usuario que sea fácil de usar garantizando la
          experiencia de usuario óptima.
\end{enumerate}

Con estos objetivos, salir un día por cuenta propia o con amigos en un entorno desconocido será mucho más
sencillo. La aplicación permitirá organizar salidas, encontrar los mejores lugares de ocio y disfrutar de
ofertas y eventos sin la necesidad de navegar por múltiples páginas web. Además, la integración de
funcionalidades sociales y de reseñas enriquecerá la experiencia del usuario, haciendo de cada salida una
experiencia agradable y bien planificada.

