\chapter{Conclusiones y Trabajo Futuro}

\section{Conclusión}

A lo largo de este proyecto, se han abordado desafíos cruciales enfrentados por los jóvenes y turistas que exploran ciudades desconocidas, especialmente aquellos relacionados con la localización de ocio y entretenimiento. La solución propuesta ha demostrado ser una herramienta valiosa, integrando diversas funcionalidades diseñadas para mejorar la eficiencia y la satisfacción de los usuarios en sus actividades sociales y de ocio. Esta plataforma no solo centraliza información relevante y actualizada de eventos y establecimientos sino que también promueve la interacción social y la participación activa de la comunidad.

El desarrollo de \textbf{HangOut} ha contribuido significativamente a resolver la problemática de la fragmentación de la información y la dificultad en la gestión de eventos. Al proporcionar una lista centralizada y personalizada de opciones de ocio, basada en las preferencias del usuario, ha permitido una exploración más intuitiva y un descubrimiento más eficaz de oportunidades de ocio.

El presente Trabajo de Fin de Grado se ha completado satisfactoriamente con el desarrollo de una aplicación móvil diseñada para la búsqueda, gestión y promoción de establecimientos orientados al ocio. Dicha herramienta ha logrado cumplir con los objetivos establecidos al inicio del proyecto, incorporando además diversas funcionalidades que enriquecen la interacción y mejoran la experiencia de los usuarios.

Este proceso ha resultado ser beneficioso tanto a nivel personal como profesional. El diseño y desarrollo de la aplicación desde su inicio me ha brindado la oportunidad de aplicar los conocimientos adquiridos a lo largo del grado. Mediante la implementación de diversas tecnologías y metodologías, he fortalecido mi comprensión en áreas previamente conocidas y he explorado otras que anteriormente había tratado superficialmente.

La selección de las tecnologías discutidas en capítulos anteriores ha ampliado significativamente mi comprensión sobre el desarrollo de aplicaciones, llevándome a alcanzar un nivel de profundidad y práctica que anteriormente no había experimentado. Adicionalmente, la implementación de metodologías de desarrollo ha potenciado mis habilidades en planificación y gestión de proyectos, enseñándome la importancia de adaptarse a los cambios dentro de un ciclo de desarrollo.

En conclusión, este proyecto ha alcanzado exitosamente su objetivo de facilitar la conexión entre usuarios y establecimientos. Adicionalmente, ha reforzado mi capacidad para identificar problemas y diseñar soluciones efectivas, contribuyendo significativamente a mi desarrollo profesional y preparándome de manera óptima para enfrentar los desafíos del mundo real en la ingeniería de software. Además, ha establecido una base sólida para futuras mejoras y desarrollos en este campo.

Cabe destacar que este proyecto es de código abierto y su código fuente se encuentra disponible en el repositorio de GitHub:

\url{https://github.com/AlexColladodev/HangOut.git}.

\noindent Esto permite a otros desarrolladores y entusiastas del software contribuir, mejorar y adaptar la aplicación según sus necesidades.


\section{Trabajo Futuro}

De cara al futuro, se identifican diversas líneas de trabajo mediante las cuales el proyecto podría expandirse, considerando que actualmente el prototipo ha alcanzado el 80\% de su desarrollo. A continuación, se presentan algunas actualizaciones para el trabajo futuro que podrían ser consideradas para completar la aplicación:

\begin{enumerate}
    \item \textbf{Desplegar API: }El despliegue efectivo de la API es esencial para garantizar su accesibilidad y constante funcionamiento. Esta tarea debería priorizarse como una de las primeras mejoras al proyecto, consiste en la configuración de un servidor que sea capaz de gestionar las solicitudes de los usuarios de manera eficiente y segura. Es fundamental que dicho servidor ofrezca escalabilidad para adaptarse al incremento en el número de usuarios, asegurando así el mantenimiento de su rendimiento óptimo.
    \item \textbf{Chat en Tiempo Real: }La implementación de un chat en tiempo real dentro de la aplicación facilitará la comunicación instantánea entre usuarios, eliminando la necesidad de utilizar otras redes sociales para este propósito. Esta funcionalidad no solo mejorará la interacción social, sino que también permitirá organizar actividades de manera más efectiva, adaptando las fechas y horarios a los requerimientos específicos de los usuarios.
    \item \textbf{Galería Común: }La inclusión de una galería común en la aplicación ofrecería a los usuarios un espacio dedicado para compartir fotos y vídeos de sus experiencias en eventos o actividades. Esta funcionalidad facilitaría que todos los usuarios accedan a las fotos de su interés y las descarguen directamente, eliminando la necesidad de solicitarlas a otros participantes.
    \item \textbf{Integración con Google Maps: }Integrar Google Maps permitirá ofrecer funcionalidades avanzadas de localización, como mostrar la ubicación de los establecimientos o especificar la ubicación de donde se realizan las actividades. Esto mejoraría significativamente la usabilidad de la aplicación.
    \item \textbf{Validación de Datos Administrador: }Implementar un sistema de validación robusto para los datos ingresados por los administradores de los establecimientos asegurando que la información proporcionada sea fiable y actualizada.
\end{enumerate}

En resumen, estas mejoras y actualizaciones propuestas están diseñadas para optimizar la experiencia del usuario y continuar el desarrollo del proyecto hacia un producto más completo y funcional.