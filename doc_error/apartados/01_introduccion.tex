\chapter{Introducción}
\section{Contexto y motivación}

Se estima que el 60 \% de los estudiantes de grado de la Universidad de Granada provienen de otras ciudades de
España o del extranjero \cite{ugr}. Muchos de ellos llegan por primera vez careciendo de información básica sobre qué
establecimientos visitar y qué hacer durante su estancia. Son personas jóvenes con intenciones de crear vínculos
y grupos con otros sobre sus aficiones y gustos compartidos.

Además de ser una ciudad universitaria, Granada es una ciudad histórica que atrae a muchos turistas que vienen
por primera vez con la emoción de descubrir este rincón de España. Estos turistas enfrentan el obstáculo de
tener que buscar entre todos los sitios disponibles hasta encontrar el más adecuado a sus preferencias.

Esta situación no es exclusiva de Granada; es un problema común en muchas ciudades del mundo, donde la falta de
información sobre qué hacer o dónde ir se convierte en un desafío habitual.

La elección de este tema surge de mi propia experiencia al llegar a Granada por primera vez. En un entorno nuevo
y desconocido, sin conocer a nadie y habiendo hecho amigos en la universidad, aún no sabía a qué lugares ir,
cuál era el ambiente de esos sitios ni qué tipo de personas los frecuentaban.

Frente a este escenario, el propósito del proyecto es el desarrollo de una aplicación móvil que asista a las
personas en la búsqueda de lugares o eventos de ocio, ofreciendo una lista de establecimientos y eventos. De
esta forma, se busca garantizar una experiencia de usuario óptima, permitiendo a los usuarios explorar
descripciones detalladas de los establecimientos según el ambiente que ofrecen, y realizar búsquedas
personalizadas según sus preferencias, sin necesidad de revisar individualmente cada página web de los
establecimientos.


\section{Objetivos}

Dado este enfoque, se han analizado y planteado una serie de objetivos para la creación de una aplicación móvil
que ofrezca una solución nueva y eficiente para la gestión y búsqueda de establecimientos orientados al ocio. El
objetivo principal es proporcionar al usuario una herramienta que facilite este proceso.

Para la consecución de este objetivo general se plantean los siguientes objetivos específicos:

\begin{enumerate}
      \item \textbf{Estudio de Tecnologías Actuales: }Realizar un análisis exhaustivo de las tecnologías y herramientas de código abierto más avanzadas disponibles para el desarrollo de aplicaciones móviles. Este estudio garantizará la selección de soluciones que no solo proporcionen la mayor eficiencia, seguridad y compatibilidad para la plataforma, sino que también ofrezcan ventajas como mayor transparencia, flexibilidad y reducción de costos asociados con licencias.
      \item \textbf{Plantear de un Diseño Escalable: }Desarrollar una arquitectura de software que no solo atienda las necesidades actuales sino que también permita futuras expansiones y modificaciones sin comprometer la funcionalidad ni el rendimiento de la aplicación.
      \item \textbf{Facilitar la Búsqueda de Establecimientos:} Desarrollar un sistema que permita a los usuarios encontrar establecimientos de ocio según sus preferencias personales y aquellos mejor valorados, basándose en las reseñas de otros usuarios. Implementar un método de ordenación eficaz que priorice los establecimientos con mayor cantidad de reseñas, asegurando así la validez y la confiabilidad de las recomendaciones ofrecidas.
      \item \textbf{Gestión de Eventos y Ofertas:} Permitir la creación, borrado y modificación de eventos y
            ofertas para que los establecimientos puedan anunciar información que le interese al usuario.
      \item \textbf{Interacción Social:} Los usuarios pueden seguir a otros, crear actividades y organizar eventos
            grupales privados.
      \item \textbf{Reseñas a Establecimientos:} Los usuarios pueden calificar la experiencia en un
            establecimiento dejando reseñas con una calificación y un mensaje
      \item \textbf{Interfaz Intuitiva:} Desarrollar una interfaz de usuario que sea fácil de usar garantizando la
            experiencia de usuario óptima.
      \item \textbf{Desplegar en un Entorno Local: }Configurar y desplegar la aplicación en un entorno local con el objetivo de probar su correcto funcionamiento. Este enfoque permite realizar una serie de pruebas integrales que aseguran la robustez y la estabilidad de la aplicación.
\end{enumerate}


